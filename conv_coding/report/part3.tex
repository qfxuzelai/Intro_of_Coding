\part{个人总结}

在本次编程实验中,我主要负责完成了:

\begin{itemize}
    \item 场景一下BMPSK映射的设计、推导、实现、仿真、折衷,对应的代码文件为
    \begin{itemize}
        \item BMPSK\_mod.m
        \item BMPSK\_demod.m
        \item BMPSK\_sim.m
        \item BMPSK\_tradeoff.m
    \end{itemize}
    \item 卷积码编译码的编码、硬判决、软判决、仿真,对应的代码文件为
    \begin{itemize}
        \item conv\_encode.m
        \item conv\_hdecode.m
        \item conv\_sdecode.m
        \item conv\_sim.m
    \end{itemize}
\end{itemize}

本次编程实验对于我和队友都是第一次初步实现一个通信实验平台,也让我对各编译码模块的工作原理和整体的通信框架有了更进一步的认识。

在BMPSK模块的设计中,我通过对信道特性的分析将MPSK映射方式进行了引入直流偏置改造,并利用改造后的映射方式推导出了对信道相位的估计方法,最后通过对偏置系数的折衷,在仿真中表现出了不错的性能。这其中让我收获最大的,是对于BMPSK参数的折衷分析:通过对映射方式性能不同角度的考量,我们对BMPSK的偏置系数提出了不同的需求,而通过在仿真中对该参数进行折衷处理,我们最终获得了较为理想的判决性能。这也让我认识到在实际通信系统的设计中有许多需要我们综合考量的指标,为了获得最优的性能,我们需要从全局的角度出发对系统参数进行设计。

在卷积码编译码模块的设计中,我根据理论课知识实现了卷积码编码模块,硬判决模块和软判决模块,这主要锻炼了我的代码实现能力。这其中让我收获最大的,是在软判决中针对多步状态转移实现的联合Viterbi译码:在这一部分的实现中,我将理论课上单步状态转移的Viterbi译码推广为了多步状态转移的联合Viterbi译码,并在软判决中使用了统一的算法框架进行实现,这一过程较好地锻炼了我进行算法设计和代码实现的能力。
